\documentclass{beamer}

\usepackage[T1]{fontenc}
\usepackage[utf8]{inputenc}
\usepackage[portuges]{babel}
\usepackage{amssymb,amsmath,amsrefs}
\usepackage{hyperref}
\usepackage{graphicx}

\newcommand{\p}{\partial}

\title{Escrita colaborativa com Git}
\author[KONZEN,AZEVEDO,GUIDI,JUSTO,SAUTER]{P.H.A.~Konzen \and F.S.~Azevedo \and L.F.~Guidi \and D.A.R.~Justo \and E.~Sauter}
\institute[IME-UFRGS]{Instituto de Matemática e Estatística\\
Universidade Federal do Rio Grande do Sul}
\date[XVIII Salão de Extensão]{XVIII Salão de Extensão\\16-19 de outubro de 2017\\Porto Alegre, RS\\UFRGS}

\begin{document}

\frame{\titlepage}

\begin{frame}{Sumário}
  \tableofcontents
\end{frame}

\section{Introdução}
\begin{frame}{Introdução}
  \begin{center}
    Escrita colaborativa é a criação de textos de modo colaborativo, supervisionado ou não por um grupo de organizadores~\cite{Wiki2017a}.
  \end{center}
  \begin{itemize}
  \item Exemplos de projetos:
    \begin{itemize}
    \item Wikipédia
    \item Cursos da Software Carpentry
    \item Projetos de ficção colaborativa:
      \begin{itemize}
      \item  Carvens, A Million Penguins, ...
      \end{itemize}
    \end{itemize}
  \end{itemize}
\vspace{0.1cm}
\begin{minipage}[hb!]{1.0\linewidth}
\hrule
\begin{bibdiv}
  \begin{biblist*}
    \bib{Wiki2017a}{misc}{
      author = {Wikipédia},
      title = {Escrita colaborativa --- Wikipédia{,} a enciclopédia livre},
      year = {2017},
      url = {\url{https://pt.wikipedia.org/w/index.php?title=Escrita_colaborativa&oldid=49878436}},
      note = {[Online; accessed 17-setembro-2017]}
    }
  \end{biblist*}
\end{bibdiv}
\end{minipage}
\end{frame}

\end{document}